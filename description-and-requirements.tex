\documentclass[a4paper]{report}
\usepackage[ngerman]{babel}

%%%%%%%%%%%%%%%%%%%%%%%%%%%%%%%%%%%%%%%%%
% Uppsala University Assignment Title Page 
% LaTeX Template
% Version 1.0 (27/12/12)
%
% This template has been downloaded from:
% http://www.LaTeXTemplates.com
%
% Original author:
% WikiBooks (http://en.wikibooks.org/wiki/LaTeX/Title_Creation)
% Modified by Elsa Slattegard to fit Uppsala university
% License:
% CC BY-NC-SA 3.0 (http://creativecommons.org/licenses/by-nc-sa/3.0/)

%\title{Title page with logo}
%----------------------------------------------------------------------------------------
%	PACKAGES AND OTHER DOCUMENT CONFIGURATIONS
%----------------------------------------------------------------------------------------


\usepackage[utf8x]{inputenc}
\usepackage{amsmath}
\usepackage{graphicx}
\usepackage{float}
\usepackage[colorinlistoftodos]{todonotes}

\begin{document}

\begin{titlepage}

\newcommand{\HRule}{\rule{\linewidth}{0.5mm}} % Defines a new command for the horizontal lines, change thickness here

\center % Center everything on the page
 
%----------------------------------------------------------------------------------------
%	HEADING SECTIONS
%----------------------------------------------------------------------------------------

%\textsc{\LARGE ISFATES - DFHI}\\[1.5cm] % Name of your university/college
\includegraphics[scale=.6]{2844169_mxtuSNlblctu_0.png}\\[1cm] % Include a department/university logo - this will require the graphicx package
\textsc{\Large Informatique et ingénierie du web (L3)}\\[0.5cm] % Major heading such as course name
\textsc{\large Perfectionnement en Java}\\[0.5cm] % Minor heading such as course title

%----------------------------------------------------------------------------------------
%	TITLE SECTION
%----------------------------------------------------------------------------------------

\HRule \\[0.4cm]
{ \huge \bfseries Wetterstation}\\[0.4cm] % Title of your document
\HRule \\[1.5cm]
 
%----------------------------------------------------------------------------------------
%	AUTHOR SECTION
%----------------------------------------------------------------------------------------

\begin{minipage}{0.4\textwidth}
\begin{flushleft} \large
%\emph{Author:}\\
Julian \textsc{Krug}\\ % Your name
Hanna \textsc{Schall}\\ % Your name
\end{flushleft}

\end{minipage}\\[2cm]

% If you don't want a supervisor, uncomment the two lines below and remove the section above
%\Large \emph{Author:}\\
%John \textsc{Smith}\\[3cm] % Your name

%----------------------------------------------------------------------------------------
%	DATE SECTION
%----------------------------------------------------------------------------------------

{\large Oktober 2020}\\[2cm] % Date, change the \today to a set date if you want to be precise

\vfill % Fill the rest of the page with whitespace

\end{titlepage}


\chapter{Projektbeschreibung}
\chapter{Anforderungsanalyse}
Im Folgenden werden die Anforderungen an das Projekt analysiert.

\section{Funktionale Anforderungen (functional requirements)}
Die funktionalen Anforderungen sind gruppiert in die Bereiche \textit{Wetterstation} und \textit{Sensor}.
\subsection{Sensor}
\begin{description}
    \item[FR1] Der Sensor bietet einen Info-Dienst, der das Senden von ProductId, VendorId und Standort beinhaltet, an.
    \item[FR2] Der Sensor bietet einen Data-Dienst, der periodisch Zeit und aktuelle Temperatur an die Wetterstation schickt, an.
    \item[FR3] Die Zeitangabe gibt Datum und exakte Uhrzeit (inklusive Sekunden) an.
    \item[FR3] Der Benutzer kann den Data-Dienst aktivieren und die Messfrequenz angeben.
    \item[FR4] Eine Messfrequenz wird in Sekunden angegeben und kann zwischen 1 und 120 Sekunden liegen.  
    \item[FR5] Der Sensor bietet einen Stop-Dienst, der das periodische Senden beendet, an.
    \item[FR6] Der Admin muss den Sensor über die Konsole starten.
    \item[FR7] Der Admin muss den Standort des Sensors über die Konsole eingeben.  
    \item[FR8] Wetterstation und Sensor können miteinander kommunizieren (Datenaustausch).  
\end{description}

\subsection{Wetterstation}
\begin{description}
    \item[FR9] Der Admin kann die Wetterstation über die Konsole starten.
    \item[FR10] Die Wetterstation hat eine MinMax-Funktion, die das Minimum und das Maximum der Temperatur eines Zeitraums angibt.
    \item[FR11] Der Zeitraum ist begrenzt durch den Start des Programms oder Reset von Minimum und Maximum und dem aktuellen Zeitpunkt.
    \item[FR12] Der Benutzer kann die MinMax-Funktion der Wetterstation benutzen.
    \item[FR13] Der Benutzer kann das Minimum und Maximum zurücksetzten (Reset). 
\end{description}


\section{Nicht-Funktionale Anforderungen (non-functional requirements)}
Die nicht-funktionalen Anforderungen sind gruppiert in die Bereiche \textit{Wetterstation}, \textit{Sensor} und \textit{Administrativ}.
\subsection{Sensor}
\begin{description}
    \item[NFR1] Der Sensor muss eine ProductId, eine VendorId und einen Standort haben.
    \item[NFR2] Der Sensor kann im Bereich -20 bis 50°C  messen.
    \item[NFR3] Der Sensor muss durch Unit-Tests getestet werden.
\end{description}

\subsection{Wetterstation}
\begin{description}
    \item[NFR4] Die Wetterstation kommuniziert mit dem Sensor über TCP.
    \item[NFR5] Die Wetterstation kann sich mit genau einem Sensor verbinden.
    \item[NFR6] Der Sensor muss durch Unit-Tests getestet werden.
\end{description}

\subsection{Administrativ}
\begin{description}
    \item[NFR7] Es muss ein UML-Klassendiagramm erstellt werden.
    \item[NFR8] Es muss eine Projektbeschreibung verfasst werden.
    \item[NFR9] Die Abgabefrist des Projekts ist am 02.11.2020 um 6 Uhr.
    \item[NFR10] Die Programmiersprache ist Java.
    \item[NFR11] Das Programm muss auf einem durchschnittlichen PC (mind. 1 GB RAM) lauffähig sein. 
\end{description}
\end{document}